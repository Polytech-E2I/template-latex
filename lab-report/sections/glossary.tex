\newfrenchacrdef{api}{API}{Application Programming Interface}{Spécification définissant une interface qui permet à des composants logiciels de communiquer entre eux}

\newfrenchacrdef{gui}{GUI}{Graphical User Interface}{Interface destinée à être visuellement ergonomique pour l'utilisateur d'un logiciel}

\newfrenchglsentry{pilote}{Pilote}{Composant logiciel fourni par un fabricant de périphériques pour permettre à un système d'exploitation de les utiliser correctement}

\newacrnodef{dam}{DAM}{Digigram ALP-X Manager}

\newfrenchacrdef{pcie}{PCIe}{PCI-Express}{Bus de communication série haute vitesse entre un processeur et des cartes d'extension}

\newenglishglsentry{designpattern}{Design Pattern}{motifconception}{Motif de conception}{Ensemble de << bonnes pratiques >> permettant de résoudre des problèmes communs et récurrents en programmation orientée objet}

\newglsref{decorator}{Decorator}{emballage}

\newenglishglsentry{wrapper}{Wrapper}{emballage}{Emballage}{\gls{designpattern}
qui est également parfois appelé \gls{decorator}. Il s'agit d'un objet qui en
enveloppe un autre pour modifier son comportement, tout en exposant la même
interface à l'utilisateur pour qui tout est transparent}

\newfrenchglsentry{exception}{Exception}{En programmation informatique, une exception est un évènement inattendu survenant durant l'exécution d'un programme. Si elle n'est pas gérée correctement, elle peut mener à des bugs voire des failles de sécurité.}

\newenglishglsentry{template}{Template}{patron}{Patron}{Fonctionnalité du langage C++ permettant de rendre génériques des classes ou des fonctions}

\newfrenchacrdef{uml}{UML}{Unified Modeling Language}{Méthode de représentation visuelle de l'architecture d'un logiciel}

\newenglishacrdef{se}{SE}{Système d'exploitation}{os}{OS}{Operating System}{Logiciel << racine >> gérant les ressources matérielles et logicielles d'un ordinateur. Par exemple, les plus connus pour les ordinateurs personnels sont GNU/Linux, Microsoft Windows ou encore Apple macOS}
